\documentclass{article}

\usepackage{hyperref}
\usepackage{xcolor}
\usepackage{amsmath}

\begin{document}

	\title{Leveraged Volume Sampling - Summary}
	\author{Ohad Ohayon}
	\pagenumbering{gobble}
	\maketitle
	\newpage

	\pagenumbering{arabic}

    \tableofcontents
    \newpage

    \section{Overview}
        This document summarises the contribution of the {\color{blue}\href{https://arxiv.org/abs/1802.06749}{\underline{paper}}} \textit{"Leveraged volume sampling for linear regression."}.
        It follows the structure of the paper itself, summarizing each of it's main sections.
        Finally, I will demonstrate my attempt at reproducing the results in the paper.

    \section{Introduction}
        \subsection{The Problem}
            Assume a linear regression problem where the feature vectors are the rows of an $n$ by $d$ full rank matrix $X=(x_{i}^T)$,
            the labels are the column vector $y$ and the optimal solution is $w^\ast=X^+y$.
            Define the average loss function of a solution $w$ to be $L(w)=\frac{1}{n}||Xw-y||_{2}^2$.
            In many real-world scenarios acquiring labels for some linear regression problems can be an expensive operation,
            although the regression features are readily available (as they may be just the parameters for an experiment).
            You may be accessible to the labels but you must limit the amount you request in order to deliver results within
            reasonable resource constraints.
            Assume the rows of such a sub-sampled problem to be $S$, the sub-sampled matrix and labels vector to be $X_{S}$
            and $y_{S}$ respectively and the sub-sampled solution to be $w_{S}=X_{S}^+y_{S}$.

            We'd like an effective algorithm to choose $S$, based on the features alone, whose corresponding solution $w_{S}$
            will hold the following properties with regard to the optimal solution $w^\ast$:
            \begin{itemize}
                \item $w_{S}$ is unbiased, i.e.: $E[w_{S}]=w^\ast$.
                \item The error for $w_{S}$ can be bounded by $L(w_{S}) < (1+\epsilon)L(w^\ast)$ for an $\epsilon$ of choice.
                \item Can be executed efficiently.
            \end{itemize}

            For the rest of this summary we'll assume $X$ is in general position unless otherwise specified.

        \subsection{Leverage Score Sampling}
            The leverage score $l_{i}$ of row $i$ of matrix $X=(x_{i}^T)$ is given by $l_{i}=x_{i}^T(X^TX)^{-1}x_{i}$ and those
            values sum up to $d$.
            \textit{Leverage Score Sampling} suggests selecting the multiset $S$ of $k$ rows i.i.d from $X$ with probability:

            \begin{equation}
                P(S\in{[n]^k})=\prod_{i=1}^{k}\frac{l_{i}}{d}
            \end{equation}

            With this procedure we are guaranteed\footnote{\label{leverage_score_performance_1}"Reverse iterative volume
            sampling for linear regression" P.21} that it is sufficient to choose
            $O(d\log{d}+\frac{d}{\epsilon})$ to obtain $L(w_{S}) < (1+\epsilon)L(w^\ast)$ with high probability.
            All in all this is a pretty efficient and fast algorithm.
            However, this multiplicative bound doesn't hold in expectation and the approximate solution $w_{S}$ is biased (a
            trait of i.i.d sampling algorithms). This means that you can't control for the error by repeating the process
            and averaging the estimators.

        \subsection{Volume Sampling}
            Volume sampling is currently the main alternative to Leverage Score Sampling.
            It is a non-i.i.d sampling method that attempts to sample a "diverse" subset of the rows of $X$,
            where "diversity" is in the sense of trying to maintain as much of the volume spanned by the
            columns of $X$.
            \textit{Volume Sampling} suggest selecting the set $S$ of $k$ distinct rows from $X$ with the following probability:

            \begin{equation}
                P(S\in{\binom{[n]}{k}})=\frac{det(X_{S}^TX_{S})}{\binom{n-d}{k-d}det(X^TX)}
            \end{equation}

            This resultant $w_{S}$ is unbiased, which in turn allows the averaging of multiple results to reduce the variance.
            However, there is no high probability guarantee of a $(1+\epsilon)L(w^\ast)$ bound on $L(w_{S})$ for this procedure
            (as shown in \hyperref[sec:no_mult_bound_vol]{the following section}).

    \section{No Multiplicative Bound For Standard Volume Sampling}
    \label{sec:no_mult_bound_vol}
    Placeholder text

\end{document}

